%% Generated by Sphinx.
\def\sphinxdocclass{report}
\documentclass[letterpaper,10pt,english]{sphinxmanual}
\ifdefined\pdfpxdimen
   \let\sphinxpxdimen\pdfpxdimen\else\newdimen\sphinxpxdimen
\fi \sphinxpxdimen=.75bp\relax

\PassOptionsToPackage{warn}{textcomp}
\usepackage[utf8]{inputenc}
\ifdefined\DeclareUnicodeCharacter
% support both utf8 and utf8x syntaxes
  \ifdefined\DeclareUnicodeCharacterAsOptional
    \def\sphinxDUC#1{\DeclareUnicodeCharacter{"#1}}
  \else
    \let\sphinxDUC\DeclareUnicodeCharacter
  \fi
  \sphinxDUC{00A0}{\nobreakspace}
  \sphinxDUC{2500}{\sphinxunichar{2500}}
  \sphinxDUC{2502}{\sphinxunichar{2502}}
  \sphinxDUC{2514}{\sphinxunichar{2514}}
  \sphinxDUC{251C}{\sphinxunichar{251C}}
  \sphinxDUC{2572}{\textbackslash}
\fi
\usepackage{cmap}
\usepackage[T1]{fontenc}
\usepackage{amsmath,amssymb,amstext}
\usepackage{babel}



\usepackage{times}
\expandafter\ifx\csname T@LGR\endcsname\relax
\else
% LGR was declared as font encoding
  \substitutefont{LGR}{\rmdefault}{cmr}
  \substitutefont{LGR}{\sfdefault}{cmss}
  \substitutefont{LGR}{\ttdefault}{cmtt}
\fi
\expandafter\ifx\csname T@X2\endcsname\relax
  \expandafter\ifx\csname T@T2A\endcsname\relax
  \else
  % T2A was declared as font encoding
    \substitutefont{T2A}{\rmdefault}{cmr}
    \substitutefont{T2A}{\sfdefault}{cmss}
    \substitutefont{T2A}{\ttdefault}{cmtt}
  \fi
\else
% X2 was declared as font encoding
  \substitutefont{X2}{\rmdefault}{cmr}
  \substitutefont{X2}{\sfdefault}{cmss}
  \substitutefont{X2}{\ttdefault}{cmtt}
\fi


\usepackage[Bjarne]{fncychap}
\usepackage{sphinx}

\fvset{fontsize=\small}
\usepackage{geometry}


% Include hyperref last.
\usepackage{hyperref}
% Fix anchor placement for figures with captions.
\usepackage{hypcap}% it must be loaded after hyperref.
% Set up styles of URL: it should be placed after hyperref.
\urlstyle{same}
\addto\captionsenglish{\renewcommand{\contentsname}{Main Page:}}

\usepackage{sphinxmessages}
\setcounter{tocdepth}{1}



\title{pyelastica}
\date{Mar 22, 2020}
\release{0.0.1}
\author{mattiaLab}
\newcommand{\sphinxlogo}{\vbox{}}
\renewcommand{\releasename}{Release}
\makeindex
\begin{document}

\pagestyle{empty}
\sphinxmaketitle
\pagestyle{plain}
\sphinxtableofcontents
\pagestyle{normal}
\phantomsection\label{\detokenize{index::doc}}



\chapter{elastica\sphinxhyphen{}python}
\label{\detokenize{index:elastica-python}}
\sphinxhref{https://travis-ci.com/mattialabteam}{\sphinxincludegraphics{{/Users/naughton/Documents/GitHub/elastica-python/docs/_build/doctrees/images/ca981258a3f57749a293d94ef2bb069f464c74e6/elastica-python}.svg}}

\sphinxhref{https://codecov.io/gh/mattialabteam/elastica-python}{\sphinxincludegraphics{{/Users/naughton/Documents/GitHub/elastica-python/docs/_build/doctrees/images/5002c09f580d4b7a4aaccaaea9b6180920641101/badge}.svg}}

Python version of elastica is located in this repository. Uploaded code is tested against the Timoshenko beam analytical solution. This branch contains ongoing implementation of friction and rod rod joints.

I am writting a test example


\section{Just a simple test}
\label{\detokenize{welcome_page:just-a-simple-test}}\label{\detokenize{welcome_page::doc}}
Here I will write more things


\section{Hello World}
\label{\detokenize{another_page:hello-world}}\label{\detokenize{another_page::doc}}

\subsection{Installation}
\label{\detokenize{another_page:installation}}
Install the package (or add it to your \sphinxcode{\sphinxupquote{requirements.txt}} file):

\begin{sphinxVerbatim}[commandchars=\\\{\}]
\PYG{g+gp}{\PYGZdl{}} pip install sphinx\PYGZus{}rtd\PYGZus{}theme
\end{sphinxVerbatim}

In your \sphinxcode{\sphinxupquote{conf.py}} file:


\section{Documentation}
\label{\detokenize{documentation:documentation}}\label{\detokenize{documentation::doc}}

\subsection{Rods}
\label{\detokenize{documentation:module-elastica.rod.cosserat_rod}}\label{\detokenize{documentation:rods}}\index{elastica.rod.cosserat\_rod (module)@\spxentry{elastica.rod.cosserat\_rod}\spxextra{module}}
Rod base classes and implementation details that need to be hidden from the user
\index{CosseratRod (class in elastica.rod.cosserat\_rod)@\spxentry{CosseratRod}\spxextra{class in elastica.rod.cosserat\_rod}}

\begin{fulllineitems}
\phantomsection\label{\detokenize{documentation:elastica.rod.cosserat_rod.CosseratRod}}\pysiglinewithargsret{\sphinxbfcode{\sphinxupquote{class }}\sphinxcode{\sphinxupquote{elastica.rod.cosserat\_rod.}}\sphinxbfcode{\sphinxupquote{CosseratRod}}}{\emph{n\_elements}, \emph{shear\_matrix}, \emph{bend\_matrix}, \emph{rod}, \emph{*args}, \emph{**kwargs}}{}
\end{fulllineitems}

\phantomsection\label{\detokenize{documentation:module-elastica.rod.constitutive_model}}\index{elastica.rod.constitutive\_model (module)@\spxentry{elastica.rod.constitutive\_model}\spxextra{module}}
Rod constitutive model mixins

\phantomsection\label{\detokenize{documentation:module-elastica.rod.data_structures}}\index{elastica.rod.data\_structures (module)@\spxentry{elastica.rod.data\_structures}\spxextra{module}}
Data structure wrapper for rod components


\subsection{Boundary Conditions}
\label{\detokenize{documentation:module-elastica.boundary_conditions}}\label{\detokenize{documentation:boundary-conditions}}\index{elastica.boundary\_conditions (module)@\spxentry{elastica.boundary\_conditions}\spxextra{module}}
Boundary conditions for rod
\index{FreeRod (class in elastica.boundary\_conditions)@\spxentry{FreeRod}\spxextra{class in elastica.boundary\_conditions}}

\begin{fulllineitems}
\phantomsection\label{\detokenize{documentation:elastica.boundary_conditions.FreeRod}}\pysigline{\sphinxbfcode{\sphinxupquote{class }}\sphinxcode{\sphinxupquote{elastica.boundary\_conditions.}}\sphinxbfcode{\sphinxupquote{FreeRod}}}
the base class for rod boundary conditions
also the free rod class

\end{fulllineitems}

\index{HelicalBucklingBC (class in elastica.boundary\_conditions)@\spxentry{HelicalBucklingBC}\spxextra{class in elastica.boundary\_conditions}}

\begin{fulllineitems}
\phantomsection\label{\detokenize{documentation:elastica.boundary_conditions.HelicalBucklingBC}}\pysiglinewithargsret{\sphinxbfcode{\sphinxupquote{class }}\sphinxcode{\sphinxupquote{elastica.boundary\_conditions.}}\sphinxbfcode{\sphinxupquote{HelicalBucklingBC}}}{\emph{position\_start}, \emph{position\_end}, \emph{director\_start}, \emph{director\_end}, \emph{twisting\_time}, \emph{slack}, \emph{number\_of\_rotations}}{}
boundary condition for helical buckling
controlled twisting of the ends

\end{fulllineitems}

\index{OneEndFixedRod (class in elastica.boundary\_conditions)@\spxentry{OneEndFixedRod}\spxextra{class in elastica.boundary\_conditions}}

\begin{fulllineitems}
\phantomsection\label{\detokenize{documentation:elastica.boundary_conditions.OneEndFixedRod}}\pysiglinewithargsret{\sphinxbfcode{\sphinxupquote{class }}\sphinxcode{\sphinxupquote{elastica.boundary\_conditions.}}\sphinxbfcode{\sphinxupquote{OneEndFixedRod}}}{\emph{fixed\_position}, \emph{fixed\_directors}}{}
the end of the rod fixed x{[}0{]}

\end{fulllineitems}

\phantomsection\label{\detokenize{documentation:module-elastica.external_forces}}\index{elastica.external\_forces (module)@\spxentry{elastica.external\_forces}\spxextra{module}}
External forcing for rod
\index{EndpointForces (class in elastica.external\_forces)@\spxentry{EndpointForces}\spxextra{class in elastica.external\_forces}}

\begin{fulllineitems}
\phantomsection\label{\detokenize{documentation:elastica.external_forces.EndpointForces}}\pysiglinewithargsret{\sphinxbfcode{\sphinxupquote{class }}\sphinxcode{\sphinxupquote{elastica.external\_forces.}}\sphinxbfcode{\sphinxupquote{EndpointForces}}}{\emph{start\_force}, \emph{end\_force}, \emph{ramp\_up\_time=0.0}}{}
Applies constant forces on endpoints
\index{apply\_forces() (elastica.external\_forces.EndpointForces method)@\spxentry{apply\_forces()}\spxextra{elastica.external\_forces.EndpointForces method}}

\begin{fulllineitems}
\phantomsection\label{\detokenize{documentation:elastica.external_forces.EndpointForces.apply_forces}}\pysiglinewithargsret{\sphinxbfcode{\sphinxupquote{apply\_forces}}}{\emph{system}, \emph{time=0.0}}{}
Apply forces to a system object.

In NoForces, this routine simply passes.
\begin{quote}\begin{description}
\item[{Parameters}] \leavevmode\begin{itemize}
\item {} 
\sphinxstyleliteralstrong{\sphinxupquote{system}} (\sphinxstyleliteralemphasis{\sphinxupquote{system that is Rod\sphinxhyphen{}like}}) \textendash{} 

\item {} 
\sphinxstyleliteralstrong{\sphinxupquote{time}} (\sphinxstyleliteralemphasis{\sphinxupquote{np.float}}\sphinxstyleliteralemphasis{\sphinxupquote{, }}\sphinxstyleliteralemphasis{\sphinxupquote{the time of simulation}}) \textendash{} 

\end{itemize}

\item[{Returns}] \leavevmode


\item[{Return type}] \leavevmode
None

\end{description}\end{quote}

\end{fulllineitems}


\end{fulllineitems}

\index{GravityForces (class in elastica.external\_forces)@\spxentry{GravityForces}\spxextra{class in elastica.external\_forces}}

\begin{fulllineitems}
\phantomsection\label{\detokenize{documentation:elastica.external_forces.GravityForces}}\pysiglinewithargsret{\sphinxbfcode{\sphinxupquote{class }}\sphinxcode{\sphinxupquote{elastica.external\_forces.}}\sphinxbfcode{\sphinxupquote{GravityForces}}}{\emph{acc\_gravity=array({[} 0.}, \emph{\sphinxhyphen{}9.80665}, \emph{0.     {]})}}{}
Applies a constant gravity on the entire rod
\index{apply\_forces() (elastica.external\_forces.GravityForces method)@\spxentry{apply\_forces()}\spxextra{elastica.external\_forces.GravityForces method}}

\begin{fulllineitems}
\phantomsection\label{\detokenize{documentation:elastica.external_forces.GravityForces.apply_forces}}\pysiglinewithargsret{\sphinxbfcode{\sphinxupquote{apply\_forces}}}{\emph{system}, \emph{time=0.0}}{}
Apply forces to a system object.

In NoForces, this routine simply passes.
\begin{quote}\begin{description}
\item[{Parameters}] \leavevmode\begin{itemize}
\item {} 
\sphinxstyleliteralstrong{\sphinxupquote{system}} (\sphinxstyleliteralemphasis{\sphinxupquote{system that is Rod\sphinxhyphen{}like}}) \textendash{} 

\item {} 
\sphinxstyleliteralstrong{\sphinxupquote{time}} (\sphinxstyleliteralemphasis{\sphinxupquote{np.float}}\sphinxstyleliteralemphasis{\sphinxupquote{, }}\sphinxstyleliteralemphasis{\sphinxupquote{the time of simulation}}) \textendash{} 

\end{itemize}

\item[{Returns}] \leavevmode


\item[{Return type}] \leavevmode
None

\end{description}\end{quote}

\end{fulllineitems}


\end{fulllineitems}

\index{MuscleTorques (class in elastica.external\_forces)@\spxentry{MuscleTorques}\spxextra{class in elastica.external\_forces}}

\begin{fulllineitems}
\phantomsection\label{\detokenize{documentation:elastica.external_forces.MuscleTorques}}\pysiglinewithargsret{\sphinxbfcode{\sphinxupquote{class }}\sphinxcode{\sphinxupquote{elastica.external\_forces.}}\sphinxbfcode{\sphinxupquote{MuscleTorques}}}{\emph{base\_length}, \emph{b\_coeff}, \emph{period}, \emph{wave\_number}, \emph{phase\_shift}, \emph{direction}, \emph{ramp\_up\_time=0.0}, \emph{with\_spline=False}}{}
Applies muscle torques on the body. It can apply muscle torques
as travelling wave with beta spline or only as travelling wave.
\index{apply\_torques() (elastica.external\_forces.MuscleTorques method)@\spxentry{apply\_torques()}\spxextra{elastica.external\_forces.MuscleTorques method}}

\begin{fulllineitems}
\phantomsection\label{\detokenize{documentation:elastica.external_forces.MuscleTorques.apply_torques}}\pysiglinewithargsret{\sphinxbfcode{\sphinxupquote{apply\_torques}}}{\emph{system}, \emph{time: float = 0.0}}{}
Apply torques to a Rod\sphinxhyphen{}like object.

In NoForces, this routine simply passes.
\begin{quote}\begin{description}
\item[{Parameters}] \leavevmode\begin{itemize}
\item {} 
\sphinxstyleliteralstrong{\sphinxupquote{system}} (\sphinxstyleliteralemphasis{\sphinxupquote{system that is Rod\sphinxhyphen{}like}}) \textendash{} 

\item {} 
\sphinxstyleliteralstrong{\sphinxupquote{time}} (\sphinxstyleliteralemphasis{\sphinxupquote{np.float}}\sphinxstyleliteralemphasis{\sphinxupquote{, }}\sphinxstyleliteralemphasis{\sphinxupquote{the time of simulation}}) \textendash{} 

\end{itemize}

\item[{Returns}] \leavevmode


\item[{Return type}] \leavevmode
None

\end{description}\end{quote}

\end{fulllineitems}


\end{fulllineitems}

\index{NoForces (class in elastica.external\_forces)@\spxentry{NoForces}\spxextra{class in elastica.external\_forces}}

\begin{fulllineitems}
\phantomsection\label{\detokenize{documentation:elastica.external_forces.NoForces}}\pysigline{\sphinxbfcode{\sphinxupquote{class }}\sphinxcode{\sphinxupquote{elastica.external\_forces.}}\sphinxbfcode{\sphinxupquote{NoForces}}}
Base class for external forcing for Rods

Can make this an abstract class, but its inconvenient
for the user to keep on defining apply\_forces and
apply\_torques object over and over.
\index{apply\_forces() (elastica.external\_forces.NoForces method)@\spxentry{apply\_forces()}\spxextra{elastica.external\_forces.NoForces method}}

\begin{fulllineitems}
\phantomsection\label{\detokenize{documentation:elastica.external_forces.NoForces.apply_forces}}\pysiglinewithargsret{\sphinxbfcode{\sphinxupquote{apply\_forces}}}{\emph{system}, \emph{time: float = 0.0}}{}
Apply forces to a system object.

In NoForces, this routine simply passes.
\begin{quote}\begin{description}
\item[{Parameters}] \leavevmode\begin{itemize}
\item {} 
\sphinxstyleliteralstrong{\sphinxupquote{system}} (\sphinxstyleliteralemphasis{\sphinxupquote{system that is Rod\sphinxhyphen{}like}}) \textendash{} 

\item {} 
\sphinxstyleliteralstrong{\sphinxupquote{time}} (\sphinxstyleliteralemphasis{\sphinxupquote{np.float}}\sphinxstyleliteralemphasis{\sphinxupquote{, }}\sphinxstyleliteralemphasis{\sphinxupquote{the time of simulation}}) \textendash{} 

\end{itemize}

\item[{Returns}] \leavevmode


\item[{Return type}] \leavevmode
None

\end{description}\end{quote}

\end{fulllineitems}

\index{apply\_torques() (elastica.external\_forces.NoForces method)@\spxentry{apply\_torques()}\spxextra{elastica.external\_forces.NoForces method}}

\begin{fulllineitems}
\phantomsection\label{\detokenize{documentation:elastica.external_forces.NoForces.apply_torques}}\pysiglinewithargsret{\sphinxbfcode{\sphinxupquote{apply\_torques}}}{\emph{system}, \emph{time: float = 0.0}}{}
Apply torques to a Rod\sphinxhyphen{}like object.

In NoForces, this routine simply passes.
\begin{quote}\begin{description}
\item[{Parameters}] \leavevmode\begin{itemize}
\item {} 
\sphinxstyleliteralstrong{\sphinxupquote{system}} (\sphinxstyleliteralemphasis{\sphinxupquote{system that is Rod\sphinxhyphen{}like}}) \textendash{} 

\item {} 
\sphinxstyleliteralstrong{\sphinxupquote{time}} (\sphinxstyleliteralemphasis{\sphinxupquote{np.float}}\sphinxstyleliteralemphasis{\sphinxupquote{, }}\sphinxstyleliteralemphasis{\sphinxupquote{the time of simulation}}) \textendash{} 

\end{itemize}

\item[{Returns}] \leavevmode


\item[{Return type}] \leavevmode
None

\end{description}\end{quote}

\end{fulllineitems}


\end{fulllineitems}

\index{UniformForces (class in elastica.external\_forces)@\spxentry{UniformForces}\spxextra{class in elastica.external\_forces}}

\begin{fulllineitems}
\phantomsection\label{\detokenize{documentation:elastica.external_forces.UniformForces}}\pysiglinewithargsret{\sphinxbfcode{\sphinxupquote{class }}\sphinxcode{\sphinxupquote{elastica.external\_forces.}}\sphinxbfcode{\sphinxupquote{UniformForces}}}{\emph{force}, \emph{direction=array({[}0.}, \emph{0.}, \emph{0.{]})}}{}
Applies uniform forces to entire rod
\index{apply\_forces() (elastica.external\_forces.UniformForces method)@\spxentry{apply\_forces()}\spxextra{elastica.external\_forces.UniformForces method}}

\begin{fulllineitems}
\phantomsection\label{\detokenize{documentation:elastica.external_forces.UniformForces.apply_forces}}\pysiglinewithargsret{\sphinxbfcode{\sphinxupquote{apply\_forces}}}{\emph{system}, \emph{time: float = 0.0}}{}
Apply forces to a system object.

In NoForces, this routine simply passes.
\begin{quote}\begin{description}
\item[{Parameters}] \leavevmode\begin{itemize}
\item {} 
\sphinxstyleliteralstrong{\sphinxupquote{system}} (\sphinxstyleliteralemphasis{\sphinxupquote{system that is Rod\sphinxhyphen{}like}}) \textendash{} 

\item {} 
\sphinxstyleliteralstrong{\sphinxupquote{time}} (\sphinxstyleliteralemphasis{\sphinxupquote{np.float}}\sphinxstyleliteralemphasis{\sphinxupquote{, }}\sphinxstyleliteralemphasis{\sphinxupquote{the time of simulation}}) \textendash{} 

\end{itemize}

\item[{Returns}] \leavevmode


\item[{Return type}] \leavevmode
None

\end{description}\end{quote}

\end{fulllineitems}


\end{fulllineitems}

\index{UniformTorques (class in elastica.external\_forces)@\spxentry{UniformTorques}\spxextra{class in elastica.external\_forces}}

\begin{fulllineitems}
\phantomsection\label{\detokenize{documentation:elastica.external_forces.UniformTorques}}\pysiglinewithargsret{\sphinxbfcode{\sphinxupquote{class }}\sphinxcode{\sphinxupquote{elastica.external\_forces.}}\sphinxbfcode{\sphinxupquote{UniformTorques}}}{\emph{torque}, \emph{direction=array({[}0.}, \emph{0.}, \emph{0.{]})}}{}
Applies uniform torque to entire rod
\index{apply\_torques() (elastica.external\_forces.UniformTorques method)@\spxentry{apply\_torques()}\spxextra{elastica.external\_forces.UniformTorques method}}

\begin{fulllineitems}
\phantomsection\label{\detokenize{documentation:elastica.external_forces.UniformTorques.apply_torques}}\pysiglinewithargsret{\sphinxbfcode{\sphinxupquote{apply\_torques}}}{\emph{system}, \emph{time: float = 0.0}}{}
Apply torques to a Rod\sphinxhyphen{}like object.

In NoForces, this routine simply passes.
\begin{quote}\begin{description}
\item[{Parameters}] \leavevmode\begin{itemize}
\item {} 
\sphinxstyleliteralstrong{\sphinxupquote{system}} (\sphinxstyleliteralemphasis{\sphinxupquote{system that is Rod\sphinxhyphen{}like}}) \textendash{} 

\item {} 
\sphinxstyleliteralstrong{\sphinxupquote{time}} (\sphinxstyleliteralemphasis{\sphinxupquote{np.float}}\sphinxstyleliteralemphasis{\sphinxupquote{, }}\sphinxstyleliteralemphasis{\sphinxupquote{the time of simulation}}) \textendash{} 

\end{itemize}

\item[{Returns}] \leavevmode


\item[{Return type}] \leavevmode
None

\end{description}\end{quote}

\end{fulllineitems}


\end{fulllineitems}

\phantomsection\label{\detokenize{documentation:module-elastica.interaction}}\index{elastica.interaction (module)@\spxentry{elastica.interaction}\spxextra{module}}
Interaction module
\index{AnistropicFrictionalPlane (class in elastica.interaction)@\spxentry{AnistropicFrictionalPlane}\spxextra{class in elastica.interaction}}

\begin{fulllineitems}
\phantomsection\label{\detokenize{documentation:elastica.interaction.AnistropicFrictionalPlane}}\pysiglinewithargsret{\sphinxbfcode{\sphinxupquote{class }}\sphinxcode{\sphinxupquote{elastica.interaction.}}\sphinxbfcode{\sphinxupquote{AnistropicFrictionalPlane}}}{\emph{k}, \emph{nu}, \emph{plane\_origin}, \emph{plane\_normal}, \emph{slip\_velocity\_tol}, \emph{static\_mu\_array}, \emph{kinetic\_mu\_array}}{}~\index{apply\_forces() (elastica.interaction.AnistropicFrictionalPlane method)@\spxentry{apply\_forces()}\spxextra{elastica.interaction.AnistropicFrictionalPlane method}}

\begin{fulllineitems}
\phantomsection\label{\detokenize{documentation:elastica.interaction.AnistropicFrictionalPlane.apply_forces}}\pysiglinewithargsret{\sphinxbfcode{\sphinxupquote{apply\_forces}}}{\emph{system}, \emph{time=0.0}}{}
Apply forces to a system object.

In NoForces, this routine simply passes.
\begin{quote}\begin{description}
\item[{Parameters}] \leavevmode\begin{itemize}
\item {} 
\sphinxstyleliteralstrong{\sphinxupquote{system}} (\sphinxstyleliteralemphasis{\sphinxupquote{system that is Rod\sphinxhyphen{}like}}) \textendash{} 

\item {} 
\sphinxstyleliteralstrong{\sphinxupquote{time}} (\sphinxstyleliteralemphasis{\sphinxupquote{np.float}}\sphinxstyleliteralemphasis{\sphinxupquote{, }}\sphinxstyleliteralemphasis{\sphinxupquote{the time of simulation}}) \textendash{} 

\end{itemize}

\item[{Returns}] \leavevmode


\item[{Return type}] \leavevmode
None

\end{description}\end{quote}

\end{fulllineitems}


\end{fulllineitems}

\index{SlenderBodyTheory (class in elastica.interaction)@\spxentry{SlenderBodyTheory}\spxextra{class in elastica.interaction}}

\begin{fulllineitems}
\phantomsection\label{\detokenize{documentation:elastica.interaction.SlenderBodyTheory}}\pysiglinewithargsret{\sphinxbfcode{\sphinxupquote{class }}\sphinxcode{\sphinxupquote{elastica.interaction.}}\sphinxbfcode{\sphinxupquote{SlenderBodyTheory}}}{\emph{dynamic\_viscosity}}{}~\index{apply\_forces() (elastica.interaction.SlenderBodyTheory method)@\spxentry{apply\_forces()}\spxextra{elastica.interaction.SlenderBodyTheory method}}

\begin{fulllineitems}
\phantomsection\label{\detokenize{documentation:elastica.interaction.SlenderBodyTheory.apply_forces}}\pysiglinewithargsret{\sphinxbfcode{\sphinxupquote{apply\_forces}}}{\emph{system}, \emph{time=0.0}}{}
This function applies hydrodynamic forces on body
using the slender body theory given in
Eq. 4.13 Gazzola et. al. RSoS 2018 paper
\begin{quote}\begin{description}
\item[{Parameters}] \leavevmode
\sphinxstyleliteralstrong{\sphinxupquote{system}} \textendash{} 

\end{description}\end{quote}

\end{fulllineitems}


\end{fulllineitems}

\index{find\_slipping\_elements() (in module elastica.interaction)@\spxentry{find\_slipping\_elements()}\spxextra{in module elastica.interaction}}

\begin{fulllineitems}
\phantomsection\label{\detokenize{documentation:elastica.interaction.find_slipping_elements}}\pysiglinewithargsret{\sphinxcode{\sphinxupquote{elastica.interaction.}}\sphinxbfcode{\sphinxupquote{find\_slipping\_elements}}}{\emph{velocity\_slip}, \emph{velocity\_threshold}}{}
This function takes the velocity of elements and checks if they are larger
than the threshold velocity. If velocity of elements are larger than
threshold velocity, that means those elements are slipping, in other words
kinetic friction will be acting on those elements not static friction. This
function output an array called slip function, this array has a size of number
of elements. If velocity of element is smaller than the threshold velocity slip
function value for that element is 1, which means static friction is acting on
that element. If velocity of element is larger than the threshold velocity slip
function value for that element is between 0 and 1, which means kinetic friction
is acting on that element.
\begin{quote}\begin{description}
\item[{Parameters}] \leavevmode\begin{itemize}
\item {} 
\sphinxstyleliteralstrong{\sphinxupquote{velocity\_slip}} \textendash{} 

\item {} 
\sphinxstyleliteralstrong{\sphinxupquote{velocity\_threshold}} \textendash{} 

\end{itemize}

\item[{Returns}] \leavevmode


\item[{Return type}] \leavevmode
slip function

\end{description}\end{quote}

\end{fulllineitems}

\index{node\_to\_element\_velocity (in module elastica.interaction)@\spxentry{node\_to\_element\_velocity}\spxextra{in module elastica.interaction}}

\begin{fulllineitems}
\phantomsection\label{\detokenize{documentation:elastica.interaction.node_to_element_velocity}}\pysigline{\sphinxcode{\sphinxupquote{elastica.interaction.}}\sphinxbfcode{\sphinxupquote{node\_to\_element\_velocity}}}
This function computes to velocity on the elements.
Here we define a seperate function because benchmark results
showed that using numba, we get almost 3 times faster calculation
\begin{quote}\begin{description}
\item[{Parameters}] \leavevmode
\sphinxstyleliteralstrong{\sphinxupquote{node\_velocity}} \textendash{} 

\item[{Returns}] \leavevmode


\item[{Return type}] \leavevmode
element\_velocity

\end{description}\end{quote}

\begin{sphinxadmonition}{note}{Note:}
Faster than pure python for blocksize 100
python: 3.81 µs \(\pm\) 427 ns per loop (mean \(\pm\) std. dev. of 7 runs, 100000 loops each)
this version: 1.11 µs \(\pm\) 19.3 ns per loop (mean \(\pm\) std. dev. of 7 runs, 1000000 loops each)
\end{sphinxadmonition}

\end{fulllineitems}

\index{slender\_body\_forces (in module elastica.interaction)@\spxentry{slender\_body\_forces}\spxextra{in module elastica.interaction}}

\begin{fulllineitems}
\phantomsection\label{\detokenize{documentation:elastica.interaction.slender_body_forces}}\pysigline{\sphinxcode{\sphinxupquote{elastica.interaction.}}\sphinxbfcode{\sphinxupquote{slender\_body\_forces}}}
This function computes hydrodynamic forces on body using slender body theory.
Below implementation is from the Eq. 4.13 in Gazzola et. al. RSoS 2018 paper.

Fh = \sphinxhyphen{} 4*pi*mu/ln(L/r) * ((I \sphinxhyphen{} 0.5 * t\textasciigrave{}t) * v)
\begin{quote}\begin{description}
\item[{Parameters}] \leavevmode\begin{itemize}
\item {} 
\sphinxstyleliteralstrong{\sphinxupquote{tangents}} \textendash{} 

\item {} 
\sphinxstyleliteralstrong{\sphinxupquote{velocity\_collection}} \textendash{} 

\item {} 
\sphinxstyleliteralstrong{\sphinxupquote{dynamic\_viscosity}} \textendash{} 

\item {} 
\sphinxstyleliteralstrong{\sphinxupquote{length}} \textendash{} 

\item {} 
\sphinxstyleliteralstrong{\sphinxupquote{radius}} \textendash{} 

\end{itemize}

\item[{Returns}] \leavevmode
\begin{itemize}
\item {} 
\sphinxstyleemphasis{Faster than numpy einsum implementation for blocksize 100}

\item {} 
\sphinxstylestrong{numpy} (\sphinxstyleemphasis{39.5 µs \(\pm\) 6.78 µs per loop (mean \(\pm\) std. dev. of 7 runs, 10000 loops each)})

\item {} 
\sphinxstylestrong{this version} (\sphinxstyleemphasis{3.91 µs \(\pm\) 310 ns per loop (mean \(\pm\) std. dev. of 7 runs, 100000 loops each)})

\item {} 
\sphinxstyleemphasis{Unrolling loops show better performance. Also since we are working in 3D everything is}

\item {} 
\sphinxstyleemphasis{3 dimensional.}

\end{itemize}


\end{description}\end{quote}

\end{fulllineitems}

\index{sum\_over\_elements (in module elastica.interaction)@\spxentry{sum\_over\_elements}\spxextra{in module elastica.interaction}}

\begin{fulllineitems}
\phantomsection\label{\detokenize{documentation:elastica.interaction.sum_over_elements}}\pysigline{\sphinxcode{\sphinxupquote{elastica.interaction.}}\sphinxbfcode{\sphinxupquote{sum\_over\_elements}}}
This function sums all elements of input array,
using a numba jit decorator shows better performance
compared to python sum(), .sum() and np.sum()
\begin{quote}\begin{description}
\item[{Parameters}] \leavevmode
\sphinxstyleliteralstrong{\sphinxupquote{input}} \textendash{} 

\item[{Returns}] \leavevmode
\begin{itemize}
\item {} 
\sphinxstyleemphasis{Faster than sum(), .sum() and np.sum()}

\item {} 
\sphinxstyleemphasis{For blocksize = 200}

\item {} 
\sphinxstylestrong{sum()} (\sphinxstyleemphasis{36.9 µs \(\pm\) 3.99 µs per loop (mean \(\pm\) std. dev. of 7 runs, 10000 loops each)})

\item {} 
\sphinxstylestrong{.sum()} (\sphinxstyleemphasis{3.17 µs \(\pm\) 90.1 ns per loop (mean \(\pm\) std. dev. of 7 runs, 100000 loops each)})

\item {} 
\sphinxstylestrong{np.sum()} (\sphinxstyleemphasis{5.17 µs \(\pm\) 364 ns per loop (mean \(\pm\) std. dev. of 7 runs, 100000 loops each)})

\item {} 
\sphinxstylestrong{This version} (\sphinxstyleemphasis{513 ns \(\pm\) 24.6 ns per loop (mean \(\pm\) std. dev. of 7 runs, 1000000 loops each)})

\end{itemize}


\end{description}\end{quote}

\end{fulllineitems}



\subsection{Multiple Rod Connections}
\label{\detokenize{documentation:module-elastica.joint}}\label{\detokenize{documentation:multiple-rod-connections}}\index{elastica.joint (module)@\spxentry{elastica.joint}\spxextra{module}}
Joint between rods module
\index{HingeJoint (class in elastica.joint)@\spxentry{HingeJoint}\spxextra{class in elastica.joint}}

\begin{fulllineitems}
\phantomsection\label{\detokenize{documentation:elastica.joint.HingeJoint}}\pysiglinewithargsret{\sphinxbfcode{\sphinxupquote{class }}\sphinxcode{\sphinxupquote{elastica.joint.}}\sphinxbfcode{\sphinxupquote{HingeJoint}}}{\emph{k}, \emph{nu}, \emph{kt}, \emph{normal\_direction}}{}
this joint currently keeps rod one fixed and moves rod two
how couples act needs to be reconfirmed

\end{fulllineitems}



\subsection{Callback Functions}
\label{\detokenize{documentation:module-elastica.callback_functions}}\label{\detokenize{documentation:callback-functions}}\index{elastica.callback\_functions (module)@\spxentry{elastica.callback\_functions}\spxextra{module}}
Call back functions for rod
\index{CallBackBaseClass (class in elastica.callback\_functions)@\spxentry{CallBackBaseClass}\spxextra{class in elastica.callback\_functions}}

\begin{fulllineitems}
\phantomsection\label{\detokenize{documentation:elastica.callback_functions.CallBackBaseClass}}\pysigline{\sphinxbfcode{\sphinxupquote{class }}\sphinxcode{\sphinxupquote{elastica.callback\_functions.}}\sphinxbfcode{\sphinxupquote{CallBackBaseClass}}}
Base call back class, user has to derive new
call back classes from this class
\index{make\_callback() (elastica.callback\_functions.CallBackBaseClass method)@\spxentry{make\_callback()}\spxextra{elastica.callback\_functions.CallBackBaseClass method}}

\begin{fulllineitems}
\phantomsection\label{\detokenize{documentation:elastica.callback_functions.CallBackBaseClass.make_callback}}\pysiglinewithargsret{\sphinxbfcode{\sphinxupquote{make\_callback}}}{\emph{system}, \emph{time}, \emph{current\_step: int}}{}
This function will be called every time step, user can
define which parameters at which time\sphinxhyphen{}step to be called back
in derived call back class
:param system:
:type system: system is rod
:param time:
:type time: simulation time
:param current\_step:
:type current\_step: current simulation time step

\end{fulllineitems}


\end{fulllineitems}

\index{ContinuumSnakeCallBack (class in elastica.callback\_functions)@\spxentry{ContinuumSnakeCallBack}\spxextra{class in elastica.callback\_functions}}

\begin{fulllineitems}
\phantomsection\label{\detokenize{documentation:elastica.callback_functions.ContinuumSnakeCallBack}}\pysiglinewithargsret{\sphinxbfcode{\sphinxupquote{class }}\sphinxcode{\sphinxupquote{elastica.callback\_functions.}}\sphinxbfcode{\sphinxupquote{ContinuumSnakeCallBack}}}{\emph{step\_skip: int}, \emph{callback\_params}}{}
Call back function for continuum snake
\index{make\_callback() (elastica.callback\_functions.ContinuumSnakeCallBack method)@\spxentry{make\_callback()}\spxextra{elastica.callback\_functions.ContinuumSnakeCallBack method}}

\begin{fulllineitems}
\phantomsection\label{\detokenize{documentation:elastica.callback_functions.ContinuumSnakeCallBack.make_callback}}\pysiglinewithargsret{\sphinxbfcode{\sphinxupquote{make\_callback}}}{\emph{system}, \emph{time}, \emph{current\_step: int}}{}
This function will be called every time step, user can
define which parameters at which time\sphinxhyphen{}step to be called back
in derived call back class
:param system:
:type system: system is rod
:param time:
:type time: simulation time
:param current\_step:
:type current\_step: current simulation time step

\end{fulllineitems}


\end{fulllineitems}

\index{MyCallBack (class in elastica.callback\_functions)@\spxentry{MyCallBack}\spxextra{class in elastica.callback\_functions}}

\begin{fulllineitems}
\phantomsection\label{\detokenize{documentation:elastica.callback_functions.MyCallBack}}\pysiglinewithargsret{\sphinxbfcode{\sphinxupquote{class }}\sphinxcode{\sphinxupquote{elastica.callback\_functions.}}\sphinxbfcode{\sphinxupquote{MyCallBack}}}{\emph{step\_skip: int}, \emph{callback\_params}}{}
My call back class it is derived from the base call back class.
This is an example, user can use this class as an example to write
new call back classes
\index{make\_callback() (elastica.callback\_functions.MyCallBack method)@\spxentry{make\_callback()}\spxextra{elastica.callback\_functions.MyCallBack method}}

\begin{fulllineitems}
\phantomsection\label{\detokenize{documentation:elastica.callback_functions.MyCallBack.make_callback}}\pysiglinewithargsret{\sphinxbfcode{\sphinxupquote{make\_callback}}}{\emph{system}, \emph{time}, \emph{current\_step: int}}{}
This function will be called every time step, user can
define which parameters at which time\sphinxhyphen{}step to be called back
in derived call back class
:param system:
:type system: system is rod
:param time:
:type time: simulation time
:param current\_step:
:type current\_step: current simulation time step

\end{fulllineitems}


\end{fulllineitems}



\subsection{Time steppers}
\label{\detokenize{documentation:module-elastica.timestepper.symplectic_steppers}}\label{\detokenize{documentation:time-steppers}}\index{elastica.timestepper.symplectic\_steppers (module)@\spxentry{elastica.timestepper.symplectic\_steppers}\spxextra{module}}
Symplectic timesteppers and concepts
\index{PEFRL (class in elastica.timestepper.symplectic\_steppers)@\spxentry{PEFRL}\spxextra{class in elastica.timestepper.symplectic\_steppers}}

\begin{fulllineitems}
\phantomsection\label{\detokenize{documentation:elastica.timestepper.symplectic_steppers.PEFRL}}\pysigline{\sphinxbfcode{\sphinxupquote{class }}\sphinxcode{\sphinxupquote{elastica.timestepper.symplectic\_steppers.}}\sphinxbfcode{\sphinxupquote{PEFRL}}}
Position Extended Forest\sphinxhyphen{}Ruth Like Algorithm of
I.M. Omelyan, I.M. Mryglod and R. Folk, Computer Physics Communications 146, 188 (2002),
\sphinxurl{http://arxiv.org/abs/cond-mat/0110585}

\end{fulllineitems}

\index{PositionVerlet (class in elastica.timestepper.symplectic\_steppers)@\spxentry{PositionVerlet}\spxextra{class in elastica.timestepper.symplectic\_steppers}}

\begin{fulllineitems}
\phantomsection\label{\detokenize{documentation:elastica.timestepper.symplectic_steppers.PositionVerlet}}\pysigline{\sphinxbfcode{\sphinxupquote{class }}\sphinxcode{\sphinxupquote{elastica.timestepper.symplectic\_steppers.}}\sphinxbfcode{\sphinxupquote{PositionVerlet}}}
\end{fulllineitems}

\index{SymplecticLinearExponentialIntegrator (class in elastica.timestepper.symplectic\_steppers)@\spxentry{SymplecticLinearExponentialIntegrator}\spxextra{class in elastica.timestepper.symplectic\_steppers}}

\begin{fulllineitems}
\phantomsection\label{\detokenize{documentation:elastica.timestepper.symplectic_steppers.SymplecticLinearExponentialIntegrator}}\pysigline{\sphinxbfcode{\sphinxupquote{class }}\sphinxcode{\sphinxupquote{elastica.timestepper.symplectic\_steppers.}}\sphinxbfcode{\sphinxupquote{SymplecticLinearExponentialIntegrator}}}
\end{fulllineitems}

\index{SymplecticStepper (class in elastica.timestepper.symplectic\_steppers)@\spxentry{SymplecticStepper}\spxextra{class in elastica.timestepper.symplectic\_steppers}}

\begin{fulllineitems}
\phantomsection\label{\detokenize{documentation:elastica.timestepper.symplectic_steppers.SymplecticStepper}}\pysiglinewithargsret{\sphinxbfcode{\sphinxupquote{class }}\sphinxcode{\sphinxupquote{elastica.timestepper.symplectic\_steppers.}}\sphinxbfcode{\sphinxupquote{SymplecticStepper}}}{\emph{cls=None}}{}
\end{fulllineitems}



\subsection{Wrappers}
\label{\detokenize{documentation:module-elastica.wrappers.base_system}}\label{\detokenize{documentation:wrappers}}\index{elastica.wrappers.base\_system (module)@\spxentry{elastica.wrappers.base\_system}\spxextra{module}}

\subsubsection{base\_system}
\label{\detokenize{documentation:base-system}}
basic coordinating multiple, smaller systems that have an independently integrable
interface (ie. works with symplectic or explicit routines \sphinxtitleref{timestepper.py}.)
\index{BaseSystemCollection (class in elastica.wrappers.base\_system)@\spxentry{BaseSystemCollection}\spxextra{class in elastica.wrappers.base\_system}}

\begin{fulllineitems}
\phantomsection\label{\detokenize{documentation:elastica.wrappers.base_system.BaseSystemCollection}}\pysigline{\sphinxbfcode{\sphinxupquote{class }}\sphinxcode{\sphinxupquote{elastica.wrappers.base\_system.}}\sphinxbfcode{\sphinxupquote{BaseSystemCollection}}}
Base System

Technical note : We can directly subclass a list for the
most part, but this is a bad idea, as List is non abstract
\sphinxurl{https://stackoverflow.com/q/3945940}
\index{finalize() (elastica.wrappers.base\_system.BaseSystemCollection method)@\spxentry{finalize()}\spxextra{elastica.wrappers.base\_system.BaseSystemCollection method}}

\begin{fulllineitems}
\phantomsection\label{\detokenize{documentation:elastica.wrappers.base_system.BaseSystemCollection.finalize}}\pysiglinewithargsret{\sphinxbfcode{\sphinxupquote{finalize}}}{}{}
Finalizes all feature class methods

\end{fulllineitems}

\index{insert() (elastica.wrappers.base\_system.BaseSystemCollection method)@\spxentry{insert()}\spxextra{elastica.wrappers.base\_system.BaseSystemCollection method}}

\begin{fulllineitems}
\phantomsection\label{\detokenize{documentation:elastica.wrappers.base_system.BaseSystemCollection.insert}}\pysiglinewithargsret{\sphinxbfcode{\sphinxupquote{insert}}}{\emph{idx}, \emph{system}}{}
S.insert(index, value) \textendash{} insert value before index

\end{fulllineitems}


\end{fulllineitems}

\phantomsection\label{\detokenize{documentation:module-elastica.wrappers.callbacks}}\index{elastica.wrappers.callbacks (module)@\spxentry{elastica.wrappers.callbacks}\spxextra{module}}

\subsubsection{callback}
\label{\detokenize{documentation:callback}}
Provides the CallBack interface to collect data in time (see \sphinxtitleref{callback\_functions.py}).
\phantomsection\label{\detokenize{documentation:module-elastica.wrappers.connections}}\index{elastica.wrappers.connections (module)@\spxentry{elastica.wrappers.connections}\spxextra{module}}

\subsubsection{connect}
\label{\detokenize{documentation:connect}}
Provides the Connections interface to connect entities (rods,
rigid bodies) using Joints (see \sphinxtitleref{joints.py}).
\phantomsection\label{\detokenize{documentation:module-elastica.wrappers.constraints}}\index{elastica.wrappers.constraints (module)@\spxentry{elastica.wrappers.constraints}\spxextra{module}}

\subsubsection{constraints}
\label{\detokenize{documentation:constraints}}
Provides the Constraints interface to enforce boundary conditions (see \sphinxtitleref{boundary\_conditions.py}).
\phantomsection\label{\detokenize{documentation:module-elastica.wrappers.forcing}}\index{elastica.wrappers.forcing (module)@\spxentry{elastica.wrappers.forcing}\spxextra{module}}

\subsubsection{forcing}
\label{\detokenize{documentation:forcing}}
Add forces and torques to rod (external point force, b\sphinxhyphen{}spline torques etc).


\subsection{Utility Functions}
\label{\detokenize{documentation:module-elastica.transformations}}\label{\detokenize{documentation:utility-functions}}\index{elastica.transformations (module)@\spxentry{elastica.transformations}\spxextra{module}}
Rotation interface functions
\index{inv\_skew\_symmetrize() (in module elastica.transformations)@\spxentry{inv\_skew\_symmetrize()}\spxextra{in module elastica.transformations}}

\begin{fulllineitems}
\phantomsection\label{\detokenize{documentation:elastica.transformations.inv_skew_symmetrize}}\pysiglinewithargsret{\sphinxcode{\sphinxupquote{elastica.transformations.}}\sphinxbfcode{\sphinxupquote{inv\_skew\_symmetrize}}}{\emph{matrix\_collection}}{}
Safe wrapper around inv\_skew\_symmetrize that does checking
and formatting on type of matrix\_collection using format\_matrix\_shape
function.

\end{fulllineitems}

\phantomsection\label{\detokenize{documentation:module-elastica.utils}}\index{elastica.utils (module)@\spxentry{elastica.utils}\spxextra{module}}
Handy utilities
\index{isqrt (in module elastica.utils)@\spxentry{isqrt}\spxextra{in module elastica.utils}}

\begin{fulllineitems}
\phantomsection\label{\detokenize{documentation:elastica.utils.isqrt}}\pysigline{\sphinxcode{\sphinxupquote{elastica.utils.}}\sphinxbfcode{\sphinxupquote{isqrt}}}
Efficiently computes sqrt for integer values

Dropin replacement for python3.8’s isqrt function
Credits : \sphinxurl{https://stackoverflow.com/a/53983683}
\begin{quote}\begin{description}
\item[{Parameters}] \leavevmode
\sphinxstyleliteralstrong{\sphinxupquote{num}} (\sphinxstyleliteralemphasis{\sphinxupquote{int}}\sphinxstyleliteralemphasis{\sphinxupquote{, }}\sphinxstyleliteralemphasis{\sphinxupquote{input}}) \textendash{} 

\item[{Returns}] \leavevmode
\begin{itemize}
\item {} 
\sphinxstylestrong{sqrt\_num} (\sphinxstyleemphasis{int, rounded down sqrt of num})

\item {} 
\sphinxstyleemphasis{Caveats}

\item {} 
\sphinxstyleemphasis{——\sphinxhyphen{}} \textendash{}
\begin{itemize}
\item {} 
Doesn’t handle edge\sphinxhyphen{}cases of negative numbers by design

\item {} 
Doesn’t type\sphinxhyphen{}check for integers by design, although it is hinted at

\end{itemize}

\end{itemize}


\end{description}\end{quote}
\subsubsection*{Examples}

\end{fulllineitems}

\phantomsection\label{\detokenize{documentation:module-elastica._calculus}}\index{elastica.\_calculus (module)@\spxentry{elastica.\_calculus}\spxextra{module}}
Quadrature and difference kernels
\index{difference\_kernel() (in module elastica.\_calculus)@\spxentry{difference\_kernel()}\spxextra{in module elastica.\_calculus}}

\begin{fulllineitems}
\phantomsection\label{\detokenize{documentation:elastica._calculus.difference_kernel}}\pysiglinewithargsret{\sphinxcode{\sphinxupquote{elastica.\_calculus.}}\sphinxbfcode{\sphinxupquote{difference\_kernel}}}{\emph{array\_collection}}{}~\begin{quote}\begin{description}
\item[{Parameters}] \leavevmode
\sphinxstyleliteralstrong{\sphinxupquote{array\_collection}} \textendash{} 

\end{description}\end{quote}

\begin{sphinxadmonition}{note}{Note:}
Not using numpy.pad, numpy.diff, numpy.hstack for performance reasons
with pad : 23.3 µs \(\pm\) 1.65 µs per loop
without pad (previous version, see git history) : 8.3 µs \(\pm\) 195 ns per loop
without pad, hstack (this version) : 5.73 µs \(\pm\) 216 ns per loop
\begin{itemize}
\item {} 
Getting the array shape and ndim seems to add \(\pm\)0.5 µs difference

\item {} 
Diff also seems to add only \(\pm\)3.0 µs

\item {} 
As an added bonus, this works for n\sphinxhyphen{}dimensions as long as last dimension

\end{itemize}

is preserved
\end{sphinxadmonition}

\end{fulllineitems}

\index{quadrature\_kernel() (in module elastica.\_calculus)@\spxentry{quadrature\_kernel()}\spxextra{in module elastica.\_calculus}}

\begin{fulllineitems}
\phantomsection\label{\detokenize{documentation:elastica._calculus.quadrature_kernel}}\pysiglinewithargsret{\sphinxcode{\sphinxupquote{elastica.\_calculus.}}\sphinxbfcode{\sphinxupquote{quadrature\_kernel}}}{\emph{array\_collection}}{}
Simple trapezoidal quadrature rule with zero at end\sphinxhyphen{}points, in a dimension agnostic way
\begin{quote}\begin{description}
\item[{Parameters}] \leavevmode
\sphinxstyleliteralstrong{\sphinxupquote{array\_collection}} \textendash{} 

\end{description}\end{quote}

\begin{sphinxadmonition}{note}{Note:}
Not using numpy.pad, numpy.hstack for performance reasons
with pad : 23.3 µs \(\pm\) 1.65 µs per loop
without pad (previous version, see git history) : 9.73 µs \(\pm\) 168 ns per loop
without pad and hstack (this version) : 6.52 µs \(\pm\) 118 ns per loop
\begin{itemize}
\item {} 
Getting the array shape and manipulating them seems to add \(\pm\)0.5 µs difference

\item {} 
As an added bonus, this works for n\sphinxhyphen{}dimensions as long as last dimension

\end{itemize}

is preserved
\end{sphinxadmonition}

\end{fulllineitems}

\phantomsection\label{\detokenize{documentation:module-elastica._linalg}}\index{elastica.\_linalg (module)@\spxentry{elastica.\_linalg}\spxextra{module}}
Convenient linear algebra kernels
\index{levi\_civita\_tensor (in module elastica.\_linalg)@\spxentry{levi\_civita\_tensor}\spxextra{in module elastica.\_linalg}}

\begin{fulllineitems}
\phantomsection\label{\detokenize{documentation:elastica._linalg.levi_civita_tensor}}\pysigline{\sphinxcode{\sphinxupquote{elastica.\_linalg.}}\sphinxbfcode{\sphinxupquote{levi\_civita\_tensor}}}
param dim:

\end{fulllineitems}

\phantomsection\label{\detokenize{documentation:module-elastica._rotations}}\index{elastica.\_rotations (module)@\spxentry{elastica.\_rotations}\spxextra{module}}
Rotation kernels

\phantomsection\label{\detokenize{documentation:module-elastica._spline}}\index{elastica.\_spline (module)@\spxentry{elastica.\_spline}\spxextra{module}}
Spline for muscle torques acting on rod


\subsection{Systems}
\label{\detokenize{documentation:module-elastica.systems.analytical}}\label{\detokenize{documentation:systems}}\index{elastica.systems.analytical (module)@\spxentry{elastica.systems.analytical}\spxextra{module}}
Analytically integrable systems, used primarily for testing time\sphinxhyphen{}steppers
\index{SecondOrderHybridSystem (class in elastica.systems.analytical)@\spxentry{SecondOrderHybridSystem}\spxextra{class in elastica.systems.analytical}}

\begin{fulllineitems}
\phantomsection\label{\detokenize{documentation:elastica.systems.analytical.SecondOrderHybridSystem}}\pysiglinewithargsret{\sphinxbfcode{\sphinxupquote{class }}\sphinxcode{\sphinxupquote{elastica.systems.analytical.}}\sphinxbfcode{\sphinxupquote{SecondOrderHybridSystem}}}{\emph{init\_x=5.0}, \emph{init\_f=3.0}, \emph{init\_v=1.0}, \emph{init\_w=1.0}}{}~\begin{description}
\item[{Integrate a simple, non\sphinxhyphen{}linear ODE:}] \leavevmode
dx/dt = v
df/dt = \sphinxhyphen{}f * ω (f is short for frame, for lack of better notation)
dv/dt = \sphinxhyphen{}v**2
dω/dt = \sphinxhyphen{}ω**2

\end{description}

Dofs: {[}x, f, v, ω{]}, with the convention that

\_state in this case are {[}x, v, ω{]}
linear\_states are {[}f{]}
\_kin\_state are {[}x{]}, taken as a slice
\_dyn\_state are {[}v, ω{]}, taken as a slice

\end{fulllineitems}



\chapter{Indices and tables}
\label{\detokenize{index:indices-and-tables}}\begin{itemize}
\item {} 
\DUrole{xref,std,std-ref}{genindex}

\item {} 
\DUrole{xref,std,std-ref}{modindex}

\item {} 
\DUrole{xref,std,std-ref}{search}

\end{itemize}


\renewcommand{\indexname}{Python Module Index}
\begin{sphinxtheindex}
\let\bigletter\sphinxstyleindexlettergroup
\bigletter{e}
\item\relax\sphinxstyleindexentry{elastica.\_calculus}\sphinxstyleindexpageref{documentation:\detokenize{module-elastica._calculus}}
\item\relax\sphinxstyleindexentry{elastica.\_linalg}\sphinxstyleindexpageref{documentation:\detokenize{module-elastica._linalg}}
\item\relax\sphinxstyleindexentry{elastica.\_rotations}\sphinxstyleindexpageref{documentation:\detokenize{module-elastica._rotations}}
\item\relax\sphinxstyleindexentry{elastica.\_spline}\sphinxstyleindexpageref{documentation:\detokenize{module-elastica._spline}}
\item\relax\sphinxstyleindexentry{elastica.boundary\_conditions}\sphinxstyleindexpageref{documentation:\detokenize{module-elastica.boundary_conditions}}
\item\relax\sphinxstyleindexentry{elastica.callback\_functions}\sphinxstyleindexpageref{documentation:\detokenize{module-elastica.callback_functions}}
\item\relax\sphinxstyleindexentry{elastica.external\_forces}\sphinxstyleindexpageref{documentation:\detokenize{module-elastica.external_forces}}
\item\relax\sphinxstyleindexentry{elastica.interaction}\sphinxstyleindexpageref{documentation:\detokenize{module-elastica.interaction}}
\item\relax\sphinxstyleindexentry{elastica.joint}\sphinxstyleindexpageref{documentation:\detokenize{module-elastica.joint}}
\item\relax\sphinxstyleindexentry{elastica.rod.constitutive\_model}\sphinxstyleindexpageref{documentation:\detokenize{module-elastica.rod.constitutive_model}}
\item\relax\sphinxstyleindexentry{elastica.rod.cosserat\_rod}\sphinxstyleindexpageref{documentation:\detokenize{module-elastica.rod.cosserat_rod}}
\item\relax\sphinxstyleindexentry{elastica.rod.data\_structures}\sphinxstyleindexpageref{documentation:\detokenize{module-elastica.rod.data_structures}}
\item\relax\sphinxstyleindexentry{elastica.systems.analytical}\sphinxstyleindexpageref{documentation:\detokenize{module-elastica.systems.analytical}}
\item\relax\sphinxstyleindexentry{elastica.timestepper.symplectic\_steppers}\sphinxstyleindexpageref{documentation:\detokenize{module-elastica.timestepper.symplectic_steppers}}
\item\relax\sphinxstyleindexentry{elastica.transformations}\sphinxstyleindexpageref{documentation:\detokenize{module-elastica.transformations}}
\item\relax\sphinxstyleindexentry{elastica.utils}\sphinxstyleindexpageref{documentation:\detokenize{module-elastica.utils}}
\item\relax\sphinxstyleindexentry{elastica.wrappers.base\_system}\sphinxstyleindexpageref{documentation:\detokenize{module-elastica.wrappers.base_system}}
\item\relax\sphinxstyleindexentry{elastica.wrappers.callbacks}\sphinxstyleindexpageref{documentation:\detokenize{module-elastica.wrappers.callbacks}}
\item\relax\sphinxstyleindexentry{elastica.wrappers.connections}\sphinxstyleindexpageref{documentation:\detokenize{module-elastica.wrappers.connections}}
\item\relax\sphinxstyleindexentry{elastica.wrappers.constraints}\sphinxstyleindexpageref{documentation:\detokenize{module-elastica.wrappers.constraints}}
\item\relax\sphinxstyleindexentry{elastica.wrappers.forcing}\sphinxstyleindexpageref{documentation:\detokenize{module-elastica.wrappers.forcing}}
\end{sphinxtheindex}

\renewcommand{\indexname}{Index}
\printindex
\end{document}